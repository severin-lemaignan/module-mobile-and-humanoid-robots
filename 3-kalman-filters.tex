%!TEX program = xelatex

\documentclass[compress]{beamer}
%--------------------------------------------------------------------------
% Common packages
%--------------------------------------------------------------------------
\definecolor{links}{HTML}{663000}
\hypersetup{colorlinks,linkcolor=,urlcolor=links}

\usepackage[english]{babel}
\usepackage{pgfpages} % required for notes on second screen
\usepackage{graphicx}
\usepackage{longtable}
\usepackage{pgfplots}

\pgfmathdeclarefunction{gauss}{2}{%
      \pgfmathparse{1/(#2*sqrt(2*pi))*exp(-((x-#1)^2)/(2*#2^2))}%
      }

\usepackage{multicol}

\usepackage{tabularx,ragged2e}
\usepackage{booktabs}

\setlength{\emergencystretch}{3em}  % prevent overfull lines
\providecommand{\tightlist}{%
  \setlength{\itemsep}{0pt}\setlength{\parskip}{0pt}}


\usetheme{hri}
\newcommand{\source}[2]{{\tiny\it Source: \href{#1}{#2}}}

\usepackage{tikz}
\usetikzlibrary{mindmap,backgrounds,positioning}

\graphicspath{{figs/}}

\title{ROCO318 \newline Mobile and Humanoid Robots}
\subtitle{Part 3 - Kalman filters}
\date{}
\author{Séverin Lemaignan}
\institute{Centre for Neural Systems and Robotics\\{\bf Plymouth University}}

\begin{document}

\licenseframe{github.com/severin-lemaignan/[REPO]}

\maketitle

\begin{frame}{ROCO306/307 Mobile and humanoid robots}

\begin{block}{Part 3 -- Kalman filters}

\end{block}

\begin{block}{}

\end{block}

\begin{block}{See Welch and Bishop (2001) An introduction to the Kalman
filter, SIGGRAPH 2001, ACM.}

\end{block}

\begin{block}{\url{http://www.cs.unc.edu/~welch/kalman/}}

\end{block}

\begin{block}{}

\end{block}

\end{frame}

\begin{frame}{What is a Kalman filter?}

\begin{itemize}
\tightlist
\item
  Developed by Rudolph E. Kalman in 1960.
\item
  Mathematical tool that estimates the real \textbf{state} of a system
  based on uncertain sensor readings.
\item
  It assumes the system is \textbf{linear} and noise is \textbf{normal}
  (aka Gaussian).
\item
  Gives past, present and \textbf{future estimations}.
\item
  Still very effective and useful for all other classes of systems.
\item
  Hugely popular in digital control systems.
\end{itemize}

For example: speed, height, position, acceleration, \ldots{}

\end{frame}

\begin{frame}{Applications of Kalman filters}

\begin{itemize}
\tightlist
\item
  Estimating critical flight parameters for guidance of missiles.
\item
  Sensor fusion in aircraft.
\item
  Fusion of localisation estimates in GPS.
\item
  Estimating game controller sensor information.
\item
  Prediction of ball position in robot football.
\item
  Prediction of head and hands position and orientation in 3D body
  posture capture system.
\item
  Prediction of the stock market.
\item
  \ldots{}
\end{itemize}

\end{frame}

\begin{frame}{Some Kalman Filter facts}

It is a filter? Not really, it does more than filters do

\begin{itemize}
\tightlist
\item
  Taking into account sensor measurements and process variables.
\item
  Prediction forwards (and backwards if needed) in time.
\item
  No explicit frequency response
\end{itemize}

Kalman Filter is recursive

\begin{itemize}
\tightlist
\item
  It start with initial estimates and continuously updates these
  estimates according to the process model and sensor measurements
  coming in.
\end{itemize}

Highly efficient: Polynomial in measurement dimensionality \emph{k} and
state dimensionality \emph{n}: \emph{O(k2.376 + n2)}

Optimal, i.e. there is no way of doing better.

\end{frame}

\begin{frame}{Stochasticity}

The true state of a system is unknown

\begin{itemize}
\tightlist
\item
  We don't know the true speed of an air craft, or the true location of
  the robot.
\end{itemize}

This is due to stochastic (= random) noise in the measurements and the
process.

\begin{itemize}
\tightlist
\item
  Measurement noise example: the air pressure meter reading fluctuates,
  even at the same high.
\item
  Process noise example: even if we keep the accelerator in the same
  position the car never goes at exactly 60 mph.
\end{itemize}

\end{frame}

\begin{frame}{Linear system}

A linear equation is a sum of input variables

\begin{itemize}
\tightlist
\item
  For example \emph{f(x)=5x+3} is linear, \emph{f(x)=cos(x)} is not.
\item
  A linear system can be written in matrix form as \emph{} or
\end{itemize}

\begin{itemize}
\tightlist
\item
  ~
\end{itemize}

~

\end{frame}

\begin{frame}{Normal}

\begin{tikzpicture}
\begin{axis}[
  no markers, domain=-5:5, samples=100,
  axis lines*=left, xlabel=$x$, ylabel=$y$,
  every axis y label/.style={at=(current axis.above origin),anchor=south},
  every axis x label/.style={at=(current axis.right of origin),anchor=west},
  height=5cm, width=\linewidth,
  xtick={-5,...,5}, ytick={0.0,0.1,...,1.0},
  enlargelimits=false, clip=false, axis on top,
  grid = major
  ]
  %\addplot [fill=cyan!20, draw=none, domain=0:5.96] {gauss(6.5,1)} \closedcycle;
  \addplot [very thick,yellow!50!black] {gauss(0,2.24)};
  \addplot [very thick,green!50!black] {gauss(-2,0.71)};
  \addplot [very thick,red!50!black] {gauss(0,1)};
  \addplot [very thick,blue!50!black] {gauss(0,0.45)};

\end{axis}

\end{tikzpicture}

\href{http://en.wikipedia.org/wiki/Normal_distribution}{Normal} or
Gaussian

\begin{itemize}
\tightlist
\item
  Symmetrical distribution, captured with two values: mean \emph{μ} and
  variance \emph{σ2}.
\item
  Described by
\end{itemize}

\begin{itemize}
\tightlist
\item
  ~
\end{itemize}

If these would be measurement distributions of sensors, which sensor is
the best?

The factor serves to keep the integral (surface under the curve) the be
1

~

\end{frame}

\begin{frame}{Basic concepts: Gaussian or normal}

\begin{itemize}
\tightlist
\item
  The Kalman Filter assumes that measurement and process noise is normal
  (also known as Gaussian) and independent.
\end{itemize}

-s

s

m

Univariate

\textbf{m}

Multivariate

For example, a temperature sensor.

For example, the x and y position of a robot.

\end{frame}

\begin{frame}{Gaussian function (2)}

    \begin{columns}
        \begin{column}{0.4\linewidth}

  Combining Gaussians ($\mu, \sigma^2$) and ($\nu, r^2$):

            \[
                \mu' = \frac{1}{\sigma^2 + r^2}(r^2\mu + \sigma^2\nu)
            \]
            \[
                \sigma^2' = \frac{1}{\frac{1}{\sigma^2} + \frac{1}{r^2}}
            \]

            Combining two Gaussians results in a Gaussian that has a \emph{smaller standard deviation}.

        \end{column}
        \begin{column}{0.6\linewidth}

            \begin{tikzpicture}
            \begin{axis}[
            no markers, domain=-5:5, samples=100,
            axis lines*=left,
            every axis y label/.style={at=(current axis.above origin),anchor=south},
            every axis x label/.style={at=(current axis.right of origin),anchor=west},
            height=4.5cm, width=\linewidth,
            xtick={-5,-3,...,5}, ytick={0.0,0.2,...,1.0},
            enlargelimits=false, clip=false, ymax=1.2,
            grid = major
            ]
            %\addplot [fill=cyan!20, draw=none, domain=0:5.96] {gauss(6.5,1)} \closedcycle;
            \addplot [very thick,green!50!black] {gauss(-2,0.71)};
            \addplot [very thick,red!50!black] {gauss(0,1)};

            \end{axis}

            \end{tikzpicture}

            \begin{tikzpicture}
            \begin{axis}[
            no markers, domain=-5:5, samples=200,
            axis lines*=left,
            every axis y label/.style={at=(current axis.above origin),anchor=south},
            every axis x label/.style={at=(current axis.right of origin),anchor=west},
            height=4.5cm, width=\linewidth,
            xtick={-5,-3,...,5}, ytick={0.0,0.2,...,1.0},
            enlargelimits=false, clip=false, ymax=1.2,
            grid = major
            ]
            %\addplot [fill=cyan!20, draw=none, domain=0:5.96] {gauss(6.5,1)} \closedcycle;
            \addplot [very thick,green!50!black] {gauss(-2,0.71)};
            \addplot [very thick,red!50!black] {gauss(0,1)};
            \addplot [very thick,cyan!50!black] {gauss(-1.33,.34)};

            \end{axis}

            \end{tikzpicture}

        \end{column}
    \end{columns}

\end{frame}

\begin{frame}{Basic concepts: state}

The \textbf{state} of a process is a vector of real numbers capturing
the relevant information describing the process:

For example

\begin{itemize}
\tightlist
\item
  The position and speed of a wheeled robot:
\item
  The speed of a missile
\end{itemize}

\begin{itemize}
\tightlist
\item
  ~
\end{itemize}

~

\end{frame}

\begin{frame}{The basics}

The Kalman filter needs a number of parameters to run.

\begin{itemize}
\tightlist
\item
  These come from the \emph{process equations}: equations that describe
  how the state of the system in the next time step depends on the
  current state and any changes that happen to the system.
\item
  For example: a car drives down the road. Its position at time
  \emph{k+1} depends on its position at time \emph{k}, the control input
  at \emph{k} (is the car braking or accelerating) and system dynamics
  (it slows down due to friction).
\end{itemize}

\end{frame}

\begin{frame}{The process equations (1)}

\begin{itemize}
\tightlist
\item
  The process is governed by a linear difference equation
\end{itemize}

The state one time step ago

The state at time step \emph{k}

\emph{n} x \emph{n} matrix changing the previous state into the current
state

\emph{n} x \emph{l} matrix that maps the control input onto the state

Control input, a vector of size \emph{l}

The process noise, a vector of size \emph{n}.

\end{frame}

\begin{frame}{The process equations (2)}

\begin{itemize}
\tightlist
\item
  We taking \emph{m} measurements which will be related to the state
  \emph{x} according to
\end{itemize}

Measurements, a vector of size \emph{m}

\emph{m} x \emph{n} matrix mapping the state to the measurements

Measurement noise, a vector of size \emph{m}

\end{frame}

\begin{frame}{The process equations: recap of main variables}

\begin{itemize}
\tightlist
\item
  \emph{A}: matrix (\emph{n} x \emph{n}) that describes how the state
  changes from \emph{k-1 to k} without controls or noise.
\item
  \emph{B}: \emph{} matrix (\emph{n} x \emph{l}) that describes how the
  control \emph{ut} changes the state from \emph{k-1 to k.}
\item
  \emph{H}: Matrix (\emph{k} x \emph{n}) that describes how to map the
  state \emph{xk} to a measurement \emph{zk.}
\end{itemize}

\end{frame}

\begin{frame}{Covariance (1)}

\href{http://en.wikipedia.org/wiki/Covariance}{Covariance}

\begin{itemize}
\tightlist
\item
  Measure of how two variables change together.
\item
  Two series X and Y of values, each of size N.
\item
  If cov(X,Y)\textgreater{}0, then X and Y tend move together. If
  cov(X,Y)\textless{}0 then X and Y have an opposite effect on
  eachother. And cov(X,Y) = 0?
\item
  Example
\end{itemize}

\end{frame}

\begin{frame}{Covariance (2)}

\href{http://en.wikipedia.org/wiki/Covariance_matrix}{Covariance matrix}
is a matrix showing the covariance of two or more variables to each
other.

\begin{itemize}
\tightlist
\item
  If one variable changes, does the other variable change as well and in
  what direction?
\item
  Example: height, temperature and air pressure
\end{itemize}

\begin{longtable}[c]{@{}llllllll@{}}
\toprule
\begin{minipage}[t]{0.10\columnwidth}\raggedright\strut
height
\strut\end{minipage} &
\begin{minipage}[t]{0.10\columnwidth}\raggedright\strut
temperature
\strut\end{minipage} &
\begin{minipage}[t]{0.10\columnwidth}\raggedright\strut
air pressure
\strut\end{minipage} &
\begin{minipage}[t]{0.10\columnwidth}\raggedright\strut
\strut\end{minipage} &
\begin{minipage}[t]{0.10\columnwidth}\raggedright\strut
Covariance matrix
\strut\end{minipage} &
\begin{minipage}[t]{0.10\columnwidth}\raggedright\strut
\strut\end{minipage} &
\begin{minipage}[t]{0.10\columnwidth}\raggedright\strut
\strut\end{minipage} &
\begin{minipage}[t]{0.10\columnwidth}\raggedright\strut
\strut\end{minipage}\tabularnewline
\begin{minipage}[t]{0.10\columnwidth}\raggedright\strut
0
\strut\end{minipage} &
\begin{minipage}[t]{0.10\columnwidth}\raggedright\strut
20
\strut\end{minipage} &
\begin{minipage}[t]{0.10\columnwidth}\raggedright\strut
1
\strut\end{minipage} &
\begin{minipage}[t]{0.10\columnwidth}\raggedright\strut
\strut\end{minipage} &
\begin{minipage}[t]{0.10\columnwidth}\raggedright\strut
~
\strut\end{minipage} &
\begin{minipage}[t]{0.10\columnwidth}\raggedright\strut
height
\strut\end{minipage} &
\begin{minipage}[t]{0.10\columnwidth}\raggedright\strut
temp
\strut\end{minipage} &
\begin{minipage}[t]{0.10\columnwidth}\raggedright\strut
pressure
\strut\end{minipage}\tabularnewline
\begin{minipage}[t]{0.10\columnwidth}\raggedright\strut
1000
\strut\end{minipage} &
\begin{minipage}[t]{0.10\columnwidth}\raggedright\strut
10
\strut\end{minipage} &
\begin{minipage}[t]{0.10\columnwidth}\raggedright\strut
0.9
\strut\end{minipage} &
\begin{minipage}[t]{0.10\columnwidth}\raggedright\strut
\strut\end{minipage} &
\begin{minipage}[t]{0.10\columnwidth}\raggedright\strut
height
\strut\end{minipage} &
\begin{minipage}[t]{0.10\columnwidth}\raggedright\strut
2916667
\strut\end{minipage} &
\begin{minipage}[t]{0.10\columnwidth}\raggedright\strut
-29166.7
\strut\end{minipage} &
\begin{minipage}[t]{0.10\columnwidth}\raggedright\strut
-400
\strut\end{minipage}\tabularnewline
\begin{minipage}[t]{0.10\columnwidth}\raggedright\strut
2000
\strut\end{minipage} &
\begin{minipage}[t]{0.10\columnwidth}\raggedright\strut
0
\strut\end{minipage} &
\begin{minipage}[t]{0.10\columnwidth}\raggedright\strut
0.8
\strut\end{minipage} &
\begin{minipage}[t]{0.10\columnwidth}\raggedright\strut
\strut\end{minipage} &
\begin{minipage}[t]{0.10\columnwidth}\raggedright\strut
temp
\strut\end{minipage} &
\begin{minipage}[t]{0.10\columnwidth}\raggedright\strut
-29166.7
\strut\end{minipage} &
\begin{minipage}[t]{0.10\columnwidth}\raggedright\strut
291.6667
\strut\end{minipage} &
\begin{minipage}[t]{0.10\columnwidth}\raggedright\strut
4
\strut\end{minipage}\tabularnewline
\begin{minipage}[t]{0.10\columnwidth}\raggedright\strut
3000
\strut\end{minipage} &
\begin{minipage}[t]{0.10\columnwidth}\raggedright\strut
-10
\strut\end{minipage} &
\begin{minipage}[t]{0.10\columnwidth}\raggedright\strut
0.7
\strut\end{minipage} &
\begin{minipage}[t]{0.10\columnwidth}\raggedright\strut
\strut\end{minipage} &
\begin{minipage}[t]{0.10\columnwidth}\raggedright\strut
pressure
\strut\end{minipage} &
\begin{minipage}[t]{0.10\columnwidth}\raggedright\strut
-400
\strut\end{minipage} &
\begin{minipage}[t]{0.10\columnwidth}\raggedright\strut
4
\strut\end{minipage} &
\begin{minipage}[t]{0.10\columnwidth}\raggedright\strut
0.056667
\strut\end{minipage}\tabularnewline
\begin{minipage}[t]{0.10\columnwidth}\raggedright\strut
4000
\strut\end{minipage} &
\begin{minipage}[t]{0.10\columnwidth}\raggedright\strut
-20
\strut\end{minipage} &
\begin{minipage}[t]{0.10\columnwidth}\raggedright\strut
0.5
\strut\end{minipage} &
\begin{minipage}[t]{0.10\columnwidth}\raggedright\strut
\strut\end{minipage} &
\begin{minipage}[t]{0.10\columnwidth}\raggedright\strut
\strut\end{minipage} &
\begin{minipage}[t]{0.10\columnwidth}\raggedright\strut
\strut\end{minipage} &
\begin{minipage}[t]{0.10\columnwidth}\raggedright\strut
\strut\end{minipage} &
\begin{minipage}[t]{0.10\columnwidth}\raggedright\strut
\strut\end{minipage}\tabularnewline
\begin{minipage}[t]{0.10\columnwidth}\raggedright\strut
5000
\strut\end{minipage} &
\begin{minipage}[t]{0.10\columnwidth}\raggedright\strut
-30
\strut\end{minipage} &
\begin{minipage}[t]{0.10\columnwidth}\raggedright\strut
0.3
\strut\end{minipage} &
\begin{minipage}[t]{0.10\columnwidth}\raggedright\strut
\strut\end{minipage} &
\begin{minipage}[t]{0.10\columnwidth}\raggedright\strut
\strut\end{minipage} &
\begin{minipage}[t]{0.10\columnwidth}\raggedright\strut
\strut\end{minipage} &
\begin{minipage}[t]{0.10\columnwidth}\raggedright\strut
\strut\end{minipage} &
\begin{minipage}[t]{0.10\columnwidth}\raggedright\strut
\strut\end{minipage}\tabularnewline
\bottomrule
\end{longtable}

\end{frame}

\begin{frame}{The process equations (3)}

The variables \emph{wk} and \emph{vk} contain the random noise on the
state and measurements.

\begin{itemize}
\tightlist
\item
  They have a \textbf{normal} distribution.
\end{itemize}

\emph{p} stands for probability

\emph{N} is the notation for a normal distribution.

With a \textbf{covariance matrix} of \emph{Q} and \emph{R}, this
signifies the width of the normal distribution.

\end{frame}

\begin{frame}{The process equations (4)}

The goal of a Kalman filter is to \textbf{estimate} the state \emph{x}
at each time step given

\begin{itemize}
\tightlist
\item
  noisy measurements,
\item
  control input,
\item
  the process difference equation.
\end{itemize}

\end{frame}

\begin{frame}{Round and round goes the Kalman filter}

The Kalman filter continuously loops through two steps

\begin{itemize}
\tightlist
\item
  The \textbf{time update} step.
\item
  The \textbf{measurement update} step.
\end{itemize}

\end{frame}

\begin{frame}{Time and measurement update}

The \textbf{time update} uses the current state estimate and error
estimate to obtain a new estimate of the state: this is called the
\textbf{a priori} estimate

\begin{itemize}
\tightlist
\item
  ``A priori'' means that the estimate is taken before any new sensor
  measurements have come in.
\end{itemize}

The \textbf{measurement update} is run after new measurements have come
in an provide the \textbf{a posteriori} estimate.

\begin{itemize}
\tightlist
\item
  ``A posteriori'' because the estimate is made after new sensor
  measurements have come in.
\end{itemize}

\end{frame}

\begin{frame}{Time update equations}

A priori estimate of the state at time \emph{k}

A posteriori estimate of the state at time \emph{k-1}

A priori estimate of the error covariance at time \emph{k}

A poseriori estimate of the error covariance at time \emph{k-1}

\textbf{Notes}

A hat \^{} means that a variable is \textbf{estimated},

a superscript -- means the variable is \textbf{a priori}.

Process noise

\end{frame}

\begin{frame}{Measurement update equations}

The Kalman gain, this needs to calculated first

The estimated state.

Sensor

noise

\end{frame}

\begin{frame}{The Kalman filter loop}

\end{frame}

\begin{frame}{Example 1}

A Kalman filter to estimate the state of a system with \textbf{one
variable}. For this demonstration, the variable remains
\textbf{constant} (for example, measuring a voltage or a temperature).

\begin{itemize}
\tightlist
\item
  See Matlab code and demo.
\end{itemize}

\end{frame}

\begin{frame}{}

\% Our initial estimate of the system's state

x\_estimate\_posterior(1) = -0.4;

\% The recursive Kalman filter process

for k=2:COUNT

\% Time update phase

x\_estimate\_prior(k) = x\_estimate\_posterior(k-1);

P\_prior(k) = P(k-1) + Q;

\% Measurement update phase

K(k) = P\_prior(k) / ( P\_prior(k) + R );

x\_estimate\_posterior(k) = x\_estimate\_prior(k) + K(k) * ( z(k) -
x\_estimate\_prior(k) );

P(k) = (1 - K(k)) * P\_prior(k);

end

\end{frame}

\begin{frame}{}

\end{frame}

\begin{frame}{Example 2}

\begin{itemize}
\tightlist
\item
  Kalman filter to estimate a moving value (one dimensional).
\end{itemize}

\end{frame}

\begin{frame}{Example 3}

\begin{itemize}
\tightlist
\item
  Kalman filter to estimate a moving value, but sensor data becomes
  unreliable
\end{itemize}

Sensor noise shoots up between t = 70 and 90, but KF doesn't know: the Q
matrix is unchanged

Same, but KF is told that sensor noise is high, Q matrix values are
increased

\end{frame}

\begin{frame}{Extension}

In this simple example we only use a state with one variable.

\begin{itemize}
\tightlist
\item
  Most systems will have more variables (e.g. pose of a drone has 6
  states).
\item
  Most systems will have the first and second derivative in time of all
  variables (i.e. rate of change and acceleration).
\end{itemize}

\end{frame}

\begin{frame}{FAQ}

Is a Kalman Filter similar to \emph{complementary filters}?

\begin{itemize}
\tightlist
\item
  Complementary filters are often used to combine accelerometer and gyro
  readings on an IMU.
\item
  CF is simple (few lines of code, no matrices) and combines a high and
  low pass filter.
\item
  CF does not predict states into the future.
\end{itemize}

What if my problem is non-linear?

\begin{itemize}
\tightlist
\item
  There are alternative versions out there such as the Extended Kalman
  Filter (EKF) or Unscented Kalman Filter (UKF).
\end{itemize}

\url{http://www.pieter-jan.com/node/11}

\end{frame}

\begin{frame}{Further reading}

A good text on Kalman filter can be found on

\begin{itemize}
\tightlist
\item
  \url{http://www.cs.unc.edu/~welch/media/pdf/kalman_intro.pdf}
\end{itemize}

A good video lecture at Udacity

\begin{itemize}
\tightlist
\item
  Lesson 2 of \href{https://www.udacity.com/course/cs373}{Artificial
  Intelligence for Robotics at Udacity}
\end{itemize}

Video demonstrations

Kalman filter on accelerometer and gyro to read stable angle

\begin{itemize}
\tightlist
\item
  \url{http://www.youtube.com/watch?v=MJ71V_wxtuU}
\item
  \url{http://www.youtube.com/watch?v=Y3TzhXYF0Lg}
\end{itemize}

Kalman filter tracking an airplane

\begin{itemize}
\tightlist
\item
  \url{http://www.youtube.com/watch?v=0GSIKwfkFCA}
\end{itemize}

\end{frame}


\begin{frame}{}
    \begin{center}
        \Large
        That's all, folks!\\[2em]
        \normalsize
        Questions:\\
        Portland Square A216 or \url{severin.lemaignan@plymouth.ac.uk} \\[1em]

        Slides:\\ \href{https://github.com/severin-lemaignan/module-mobile-and-humanoid-robots}{\small github.com/severin-lemaignan/module-mobile-and-humanoid-robots}

    \end{center}
\end{frame}



\end{document}
